\documentclass{amsart}
\usepackage[margin=1in]{geometry}

\usepackage{amsmath,amssymb,latexsym}

\title{IMA Proposal - GeneralizedClusterAlgebra Coding Sprint}

\author{Esther Banaian}
\address[Esther Banaian]{University of Minnesota}
\email{banai003@umn.edu}
\author{Elizabeth Kelley}
\address[Elizabeth Kelley]{University of Minnesota}
\email{kell1642@umn.edu}
\author{Dylan Rupel}
\address[Dylan Rupel]{Michigan State University}
\email{dylanrupel@gmail.com}

\begin{document}

  \maketitle

  \subsection*{Overview}
  Cluster algebras are recursively defined subalgebras of affine rational function fields.
  Their ubiquity throughout mathematics and mathematical physics has warranted the creation of a ClusterAlgebra package in SAGE.
  Recently, this definition has been extended to allow polynomial (rather than binomial) exchange relations.
 This more flexible structure is working its way into mathematics, but is not currently implemented in SAGE.

  \subsection*{Goals}
  The goal of this proposal is to implement generalized cluster algebras together with their currently understood relationships to standard cluster algebras.
  Features to be implemented include:
  \begin{itemize}
    \item generalized cluster algebras;
    \item left/right companion cluster algebras;
    \item folding/unfolding (generalized) cluster algebras.
  \end{itemize}
  Additional features which may be considered upon completion of the initial project may include:
  \begin{itemize}
    \item implementing snake graphs for (generalized) cluster algebras from surfaces;
    \item reproducing ClusterComplex as a subobject of a ClusterAlgebra (implemented using g-vectors).
  \end{itemize}
  These goals align with the SAGE trac ticket \#26771.

  \subsection*{History}
  Cluster mutations first entered SAGE via the ClusterQuiver and ClusterSeed packages written by Gregg Musiker and Christian Stump.  
  Of particular importance, this code includes methods for discovering mutation classes and searching the exchange graph of a cluster algebra.
  The ClusterAlgebra package was written by the third proposer and Salvatore Stella as a reimplementation using the standard algebra/element framework.

  Generalized cluster algebras were first studied by Leonid Chekhov and Michael Shapiro for understanding surfaces with orbifold singularities.
  The original definition was simplified by Tomoki Nakanishi under certain normalization assumptions, in which case the mutation of exchange polynomials becomes particularly simple.

  \subsection*{Design}
  This initial implementation of generalized cluster algebras will likely be restricted to the normalized case.
  The proposed implementation will be achieved by modifying the current ClusterAlgebra package with a new set of parameters which recover the standard package upon specialization.
  A rough plan for the changes is:
  \begin{enumerate}
    \item add initializations for:
      \begin{itemize}
        \item exchange polynomial degrees,
        \item exchange coefficient specializations;
      \end{itemize}
    \item add global parameters representing intermediate exchange coefficients;
    \item modify matrix mutation method using exchange polynomial degrees;
    \item modify $c$-vector and $g$-vector recursions;
    \item modify $F$-polynomial recursion;
    \item implement left- and right-companion $c$-vectors, $g$-vectors, and $F$-polynomials;
    \item implement exchange coefficient specialization (this may need to be proven, c.f. Geiss, Leclerc, Schr\"oer results on quantum specialization);
    \item introduce unfold() method implementing a particular exchange coefficient specialization to the unfolded (generalized) cluster algebra;
    \item add the alias GeneralizedClusterAlgebra;
    \item add examples to the documentation.
  \end{enumerate}

  \subsection*{Participants}
  \begin{itemize}
    \item Esther Banaian is a graduate student at University of Minnesota working on (generalized) cluster algebras.
    \item Elizabeth Kelley is a graduate student at University of Minnesota working on (generalized) cluster algebras.  During SageDays@ICERM in July 2018, she contributed to the implementation of snake graphs for cluster variables from surfaces (trac ticket \#16310).
    \item Dylan Rupel is a research associate at Michigan State University.
      He has written many papers on (generalized) cluster algebras and coauthored the current ClusterAlgebra package in SAGE.
  \end{itemize}

\end{document}
